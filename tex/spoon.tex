% To do:
% ------
% - Don't forget to regularly pull and push with GitHub version.
% - Find out what MKN is writing and be complementary.
% - Write an outline.
% - Fill in the outline.
% - Publish.
% - Wait for the call from Stockholm.

% Style notes:
% ------------
% - Uhhh.

\documentclass[modern]{aastex631}
\usepackage[utf8]{inputenc}
\graphicspath{{./}{figures/}}
\usepackage{amsmath}

% Hogg typesetting issues
\addtolength{\textheight}{0.8in}
\addtolength{\topmargin}{-0.4in}
\setlength{\parindent}{1.2\baselineskip} % seriously
\frenchspacing\raggedbottom\sloppy\sloppypar

% Text issues
\newcommand{\acronym}[1]{\small{#1}}
\newcommand{\project}[1]{\textsl{#1}}
\newcommand{\Gaia}{\project{Gaia}}
\newcommand{\APOGEE}{\project{\acronym{APOGEE}}}

% Math issues
\newcommand{\teff}{T_{\mathrm{eff}}}
\newcommand{\logg}{\log g}
\newcommand{\feh}{[\mathrm{Fe}/\mathrm{H}]}

\shorttitle{machine learning and stellar parameters}
\shortauthors{apw \& dwh}

\begin{document}

\title{Data-driven models of, and with, the \textsl{Gaia} XP spectra}

\author{APW}
\author{DWH}
\author{others}

\begin{abstract}\noindent % seriously!
  Machine learning and data-driven approaches are having a huge impact in astrophysics.
  Nowhere is this more true than in the measurement of stellar parameters ($\teff$, $\logg$, $\feh$, luminosities, and detailed abundances) from spectra (fluxes as a function of wavelength, or equivalents).
  Here we discuss approaches to obtain stellar parameters from the low-resolution ESA \Gaia\ XP spectra, which are delivered in terms of coefficients in a polynomial basis.
  We compare discriminative and generative models, and we also compare simple and complex models.
  One take-away is that generative models are far more capable when the test-set object has a spectrum with low signal-to-noise ratio.
  Another is\ldots something?
\end{abstract}

\section{Introduction} \label{sec:intro}

Hello World!

\section{Regression and models}

What is a regression?

What does an ML or stats person call a ``model''?

What is generative and what is discriminative?

Why is this latter distinction important?

What is known from a math / proof perspective, especially as regards risk and correlations in the label space?

\section{Data}

\Gaia\ XP DR3 data.... What do we get and what do we do with it?

\APOGEE\ DR17 data.... What do we get and what do we do with it?

\section{Problem set-up}

Each object $i$ has a spectrum, which is written as a $r$-dimensional vector $x_i$ of spectral values or real-valued numbers.
In the case of the \Gaia\ XP spectra, these will be the spectral coefficients or ratios of the spectral coefficients, or the spectral coefficients divided by some normalization. HOGG GET SPECIFIC HERE.

Each object $i$ also has a $q$-element list or $q$-dimensional vector $y_i$ of labels, which are also (usually) real-valued numbers.
In the case of stellar-parameter label-transfer problems from \APOGEE, these might be $\teff$, $\logg$, $feh$.
In the case of luminosity or distance estimation, these might be \Gaia-measured absolute magnitudes $M_G$.

What are variances and noise models?

\section{Discriminative models}

Linear regression.

What are regularizations?

Under- and over-parameterized models.

Linear regression in neighborhoods.

Deep learning (and the like).

\section{Generative models}

The simplest generative model is the \emph{linear latent-variable model}.
In this model, both the features $x_i$ and the labels $y_i$ for object $i$ are generated by a $d$-dimensional latent variable $z_i$, which consists of $d$ real numbers:
\begin{align}
  x_i &= \mu_x + A\,z_i + \mbox{noise} \\
  y_i &= \mu_y + B\,z_i + \mbox{noise} ~,
\end{align}
where $\mu_x$ and $\mu_y$ are mean values for the $r$-vector features and the $q$-vector labels,
$A$ and $B$ are $r\times d$ and $q\times d$ rectangular matrices of coefficients,
and the ``noise'' is so far unspecified.

.... HOGG ...

\section{Unsupervised models; clustering and de-noising}

\section{Physics-inspired or hybrid models}

\section{Discussion}

Hello World!

\bibliography{sample631}{}
\bibliographystyle{aasjournal}

\end{document}
