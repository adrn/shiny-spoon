% To do:
% ------
% - Don't forget to regularly pull and push with GitHub version.
% - Find out what MKN is writing and be complementary.
% - Write an outline.
% - Fill in the outline.
% - Publish.
% - Wait for the call from Stockholm.

% Style notes:
% ------------
% - Uhhh.

\documentclass[modern]{aastex631}
\usepackage[utf8]{inputenc}
\graphicspath{{./}{figures/}}
\usepackage{amsmath}

% Hogg typesetting issues
\addtolength{\textheight}{0.8in}
\addtolength{\topmargin}{-0.4in}
\setlength{\parindent}{1.2\baselineskip} % seriously
\frenchspacing\raggedbottom\sloppy\sloppypar

% Text issues
\newcommand{\acronym}[1]{\small{#1}}
\newcommand{\project}[1]{\textsl{#1}}
\newcommand{\Gaia}{\project{Gaia}}
\newcommand{\APOGEE}{\project{\acronym{APOGEE}}}

% Math issues
\newcommand{\given}{\,|\,}
\newcommand{\teff}{T_{\mathrm{eff}}}
\newcommand{\logg}{\log g}
\newcommand{\feh}{[\mathrm{Fe}/\mathrm{H}]}

\shorttitle{machine learning and stellar parameters}
\shortauthors{apw \& dwh}

\begin{document}

\title{Data-driven models of, and with, the \textsl{Gaia} XP spectra}

\author{APW}
\author{DWH}
\author{others}

\begin{abstract}\noindent % seriously!
  Machine learning and data-driven approaches are having a huge impact in astrophysics.
  Nowhere is this more true than in the measurement of stellar parameters ($\teff$, $\logg$, $\feh$, luminosities, and detailed abundances) from spectra (fluxes as a function of wavelength, or equivalents).
  Here we discuss approaches to obtain stellar parameters from the low-resolution ESA \Gaia\ XP spectra, which are delivered in terms of coefficients in a polynomial basis.
  We compare discriminative and generative models, and we also compare simple and complex models.
  One take-away is that generative models are far more capable when the test-set object has a spectrum with low signal-to-noise ratio.
  Another is\ldots something?
\end{abstract}

\section{Introduction} \label{sec:intro}

Hello World!

\section{Regression and models}

What is a regression?

What does an ML or stats person call a ``model''?

What is generative and what is discriminative?

Why is this latter distinction important?

What is known from a math / proof perspective, especially as regards risk and correlations in the label space?

\section{Data}

\Gaia\ XP DR3 data.... What do we get and what do we do with it?

\APOGEE\ DR17 data.... What do we get and what do we do with it?

\section{Problem set-up}

Each object $i$ has a spectrum, which is written as a $r$-dimensional vector $x_i$ of spectral values or real-valued numbers.
In the case of the \Gaia\ XP spectra, these will be the spectral coefficients or ratios of the spectral coefficients, or the spectral coefficients divided by some normalization. HOGG GET SPECIFIC HERE.

Each object $i$ also has a $q$-element list or $q$-dimensional vector $y_i$ of labels, which are also (usually) real-valued numbers.
In the case of stellar-parameter label-transfer problems from \APOGEE, these might be $\teff$, $\logg$, $feh$.
In the case of luminosity or distance estimation, these might be \Gaia-measured absolute magnitudes $M_G$.

HOGG: What are variances and noise models?

HOGG: How many training and test objects are there? Perhaps $n$ and $m$?
The feature $r$-vectors $x_i$ and the label $q$-vectors $y_i$ can be usefully packed into rectangular arrays $X, Y$ such that
\begin{align}
  X^\top &= \begin{bmatrix}x_1 & x_2 & x_3 & \cdots & x_n\end{bmatrix} \\
  Y^\top &= \begin{bmatrix}y_1 & y_2 & y_3 & \cdots & y_n\end{bmatrix} ~,
\end{align}
where the transposes are shown to indicate that $X$ is $n\times r$ and $Y$ is $n\times q$.

\section{Discriminative models}

Linear regression.

What are regularizations?

Under- and over-parameterized models.

Linear regression in neighborhoods.

Deep learning (and the like).

Problems with missing features, and noisy features. SHOW that the model degrades terribly when the features get noisy.

\section{Generative models}

APW: Should we start each section with the assumptions that will dictate the model in each section?

The simplest generative model is the \emph{linear latent-variable model}.
In this model, both the features $x_i$ and the labels $y_i$ for object $i$ are generated by a $d$-dimensional latent variable $z_i$, which consists of $d$ real numbers:
\begin{align}
  x_i &= \mu_x + A\,z_i + \mbox{noise} \\
  y_i &= \mu_y + B\,z_i + \mbox{noise} ~,
\end{align}
where $\mu_x$ and $\mu_y$ are mean values for the $r$-vector features and the $q$-vector labels,
$A$ and $B$ are $r\times d$ and $q\times d$ rectangular matrices of coefficients,
and the ``noise'' is the residual.
If we think of the rectangular matrices $A, B$ as being themselves being made up of $d$-vectors, then
\begin{align}
  A^\top &= \begin{bmatrix}a_1 & a_2 & a_3 & \cdots & a_r\end{bmatrix} \\
  B^\top &= \begin{bmatrix}b_1 & b_2 & b_3 & \cdots & b_q\end{bmatrix} ~,
\end{align}
where each $d$-vector $a_j$ or $b_\ell$ is the vector whose inner product $a_j^\top z_i$ predicts (say) the $j$th component of the feature vector $x_i$ for object $i$.
Given these definitions, we can say
\begin{align}
  x_j &= Z\,a_j + \mbox{noise} \\
  y_\ell &= Z\,b_\ell + \mbox{noise} ~,
\end{align}
where now $x_j$ is the $j$th column of the data matrix $X$ (whereas $x_i$ was the $i$th row of $X$),
and now $y_\ell$ is the $\ell$th column of $Y$ (whereas $y_i$ was the $i$th row of $Y$)..

The key idea of this generative model is that the latent $d$-vectors $z_i$ contain useful information about the stellar parameters, which are among the labels $z_i$.
Thus it is useful to hard-set part of the rectangular matrix $B$ such that there are direct and simple relationships between the components of the $z_i$ vectors and the labels.
For example... HOGG....

Under the joint assumptions that the model is reasonable, the stars are independent, the noise is Gaussian, and the noise variances are known, the likelihood of this model is given by
\begin{align}
  p(X, Y\given \mu_x, A, \mu_y, B, Z) &= \prod_i p(x_i\given \mu_x, A, z_i)\,p(y_i\given \mu_y, B, z_i) \\
  p(x_i\given \mu_x, A, z_i) &= N(x_i\given \mu_x + A\,z_i, C_{xi}) \\
  p(y_i\given \mu_y, B, z_i) &= N(y_i\given \mu_y + B\,z_i, C_{yi}) ~,
\end{align}
where $Z$ is the $n\times d$ rectangular block of all $d$-vector latent variables $z_i$,
and $N(x\given\mu,V)$ is the multivariate Gaussian pdf for a vector $x$ given vector mean $\mu$ and variance tensor $V$.

At fixed $A, B$, the $z_i$ vectors can be found by weighted least squares; the problem is linear.
At fixed $Z$, $A$ and the unknown parts of $B$ (the parts not set by definition) can be found by weighted least squares.
That is, this entire linear latent-variable model can be optimized by iterated least squares, in which there is a \emph{$z$-step} in which $A, B$ are held fixed and $Z$ is optimized, followed by a \emph{$A$-step} in which $Z$ is held fixed and $A$ is optimized (plus any components of $B$ that are not set by definition).
The $z$-step can be simplified computationally by splitting the optimization of $Z$ into small optimizations by splitting the training set into objects (which are independent, by assumption), and the $A$-step can be simplified computationally similarly by splitting the optimization of $A$ into feature components (which are not independent, so WHAT HAPPENS?).

.... HOGG ...

\section{Unsupervised models; clustering and de-noising}

\section{Physics-inspired or hybrid models}

\section{Discussion}

Hello World!

\bibliography{sample631}{}
\bibliographystyle{aasjournal}

\end{document}
